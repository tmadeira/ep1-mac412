\documentclass[a4paper,oneside,12pt]{article}
\usepackage[utf8]{inputenc}
\usepackage[brazil]{babel}

\usepackage[sc]{mathpazo}
\usepackage[T1]{fontenc}
\linespread{1.3}

\title{EP I de MAC412}
\author{Ana Luísa Losnak (7240258)\\
        Tiago Madeira (6920244)}
\date{15 de setembro de 2013}

\begin{document}
\maketitle
\tableofcontents
\newpage

\section{Parte 1}

Pequena introdução (o que é a parte 1) % TODO

\subsection{Instalando o PAPI}

Iniciamos nosso trabalho baixando a biblioteca PAPI\footnote{http://icl.cs.utk.edu/papi/}, lendo seu arquivo {\tt README.txt} e seguindo algumas instruções sugeridas pelo arquivo para melhor entendermos o funcionamento da PAPI. Nesse processo, instalamos a biblioteca, acessamos a documentação do projeto e realizamos alguns testes. O sistema operacional que utilizamos foi o \textbf{Ubuntu Linux}\footnote{http://ubuntu.com/}

Para compilar e instalar o PAPI, executamos:

\begin{verbatim}
$ ./configure --prefix=/usr
$ make
$ su
# make install
\end{verbatim}

Alguns dos exemplos disponívels no diretório {\tt utils/} do projeto foram bastante importantes para entendermos a forma como a biblioteca funciona. Em particular, o utilitário {\tt utils/clockres.c} nos ajudou a entender como usar as funções para cronometrar tempos de execução reais e virtuais das operações tanto em ciclos e microssegundos.

\subsection{Contando o tempo com o PAPI}

Foram necessárias alterações muito pequenas para fazer o tempo ser cronometrado com o PAPI no lugar de {\tt gettimeofday()} nos arquivos {\tt teste1.c} e {\tt teste1\_ord.c}. Incluímos o cabeçalho {\tt papi.h} no código-fonte e a função {\tt get\_time()} foi modificada de sua versão original:

\begin{verbatim}
inline uint64_t get_time(void) {
    struct timeval t1;
    gettimeofday(&t1, NULL);
    return 1000000L * t1.tv_sec + t1.tv_usec;
}
\end{verbatim}

para:

\begin{verbatim}
inline uint64_t get_time(void) { 
    return (uint64_t) PAPI_get_virt_cyc();
}  
\end{verbatim}

A única outra necessidade foi adicionar uma chamada à função {\tt PAPI\_library\_init()} no início da função {\tt main()} para que a biblioteca fosse inicializada.

A resolução da função {\tt PAPI\_get\_virt\_cyc()} é muito maior do que a da função {\tt gettimeofday()}:

\begin{verbatim}
$ ./teste1
Time: 8171 Count 500527
$ ./teste1_ord
Time: 2671 Count 500527
$ ./teste1-papi
Time: 19567704 Count 500927
$ ./teste1_ord-papi
Time: 6495723 Count 500927
\end{verbatim}

\subsection{Hipóteses}

Pequena introdução % TODO

\subsubsection{Registradores}

Um pouco sobre hipótese 1 % TODO

\subsubsection{Contando número de falhas de cache}

Bastante sobre essa hipótese, com códigos que usamos para contar cache misses % TODO

\subsubsection{Outra hipótese}

Outra hipótese vai aqui % TODO

\section{Parte 2}

A parte 2 do EP consiste em estimar o tamanho do cache bla bla bla % TODO

\subsection{Utilizando hwloc}

Aceitamos a sugestão do enunciado do EP de usar \emph{hwloc}\footnote{http://www.open-mpi.org/software/hwloc/v1.7/} para obter os valores reais do tamanho do cache e da linha. Outras alternativas seriam ler {\tt /sys/devices/system/cpu/cpu*/*} ou ainda usar as rotinas que o PAPI no seu utilitário {\tt utils/mem\_info.c}, mas achamos interessante a promessa de um acréscimo na nota para quem usasse \emph{hwloc}.

Instalamos o programa através do gerenciador de pacotes do Ubuntu ({\tt apt-get install hwloc libhwloc-dev}). Também baixamos o código para usar como referência. Lendo o manual, descobrimos como inicializar o \emph{hwloc} e como obter os tamanhos de cache e linha:

\begin{verbatim}
/* Inicializa hwloc */
if (hwloc_topology_init(&topology)) {
    fprintf(stderr, "Erro em hwloc_topology_init\n");
    exit(1);
}
hwloc_topology_load(topology);

/* Pega tamanhos reais de cache e line */
level = 0;
for (obj = hwloc_get_obj_by_type(topology, HWLOC_OBJ_PU, 0);
        obj; obj = obj->parent) {
    if (obj->type == HWLOC_OBJ_CACHE) {
        real_cache[level] = obj->attr->cache.size;
        real_line[level] = obj->attr->cache.linesize;
        level++;
    }
}
numlevels = level;
\end{verbatim}

\subsection{Utilizando contadores do PAPI}

Havíamos aprendido a usar os contadores de hardware do PAPI na parte 1 do EP, quando formulamos a hipótese de que o programa {\tt teste1\_ord.c} seria mais rápido por causa do cache. Usamos aqui o mesmo código para contar o número de falhas de cache.

\subsection{Estimando tamanho do cache e da fila}

Aqui vai a parte difícil do EP % TODO

\section{Conclusão}

Aqui vai a conclusão % TODO

\end{document}

