\documentclass[a4paper,oneside,12pt]{article}
\usepackage[utf8]{inputenc}
\usepackage[brazil]{babel}

\usepackage{fullpage}

\usepackage[sc]{mathpazo}
\usepackage[T1]{fontenc}
\linespread{1.2}

\title{EP I de MAC412}
\author{Ana Luísa Losnak (7240258)\\
        Tiago Madeira (6920244)}
\date{15 de setembro de 2013}

\begin{document}
\maketitle
\tableofcontents
\newpage

\section{Parte 1}

A parte 1 deste EP visa a instalação da biblioteca PAPI\footnote{http://icl.cs.utk.edu/papi/} e a familiarização ao uso de suas funcionalidades, bem como a análise e discussão sobre a diferença no tempo de execução de programas similares. 

Após a instalação completa, descrita em mais detalhes abaixo, alteramos o contador de tempo dos programas disponíveis no enunciado por eventos da PAPI para maior resolução no resultado. Com esses novos programas, realizamos diversas experiências, analisando o tempo de execução destes programas e discutindo hipóteses para tal diferença. Por fim, avaliamos qual hipótese era a mais correta utilizando a biblioteca PAPI para contar diversos eventos do computador ao executar cada programa. A seguir descrevemos mais detalhadamente como foi feita a parte 1.

\subsection{Instalando o PAPI}

Iniciamos nosso trabalho baixando a biblioteca PAPI, lendo seu arquivo {\tt README.txt} e seguindo algumas instruções sugeridas pelo arquivo para melhor entendermos o funcionamento da PAPI. Nesse processo, instalamos a biblioteca, acessamos a documentação do projeto e realizamos alguns testes. O sistema operacional que utilizamos foi o \textbf{Ubuntu Linux}\footnote{http://ubuntu.com/}.

Para compilar e instalar o PAPI, executamos:

\begin{verbatim}
$ ./configure --prefix=/usr
$ make
$ su
# make install
\end{verbatim}

Alguns dos exemplos disponívels no diretório {\tt utils/} do projeto foram bastante importantes para entendermos a forma como a biblioteca funciona. Em particular, o utilitário {\tt utils/clockres.c} nos ajudou a entender como usar as funções para cronometrar tempos de execução reais e virtuais das operações tanto em ciclos como em microssegundos.

\subsection{Contando o tempo com o PAPI}

Foram necessárias alterações muito pequenas para fazer o tempo ser cronometrado com o PAPI no lugar de {\tt gettimeofday()} nos arquivos {\tt teste1.c} e {\tt teste1\_ord.c}: incluímos o cabeçalho {\tt papi.h} no código-fonte e a função {\tt get\_time()} foi modificada de sua versão original

\begin{verbatim}
inline uint64_t get_time(void) {
    struct timeval t1;
    gettimeofday(&t1, NULL);
    return 1000000L * t1.tv_sec + t1.tv_usec;
}
\end{verbatim}

para

\begin{verbatim}
inline uint64_t get_time(void) { 
    return (uint64_t) PAPI_get_virt_cyc();
}  
\end{verbatim}

A única outra necessidade foi adicionar uma chamada à função {\tt PAPI\_library\_init()} no início da função {\tt main()} para que a biblioteca fosse inicializada.

A resolução da função {\tt PAPI\_get\_virt\_cyc()} é muito maior do que a da função {\tt gettimeofday()}:

\begin{verbatim}
$ ./teste1
Time: 8171 Count 500527
$ ./teste1_ord
Time: 2671 Count 500527
$ ./teste1-papi
Time: 19567704 Count 500927
$ ./teste1_ord-papi
Time: 6495723 Count 500927
\end{verbatim}

\subsection{Hipóteses}

Com uma maior resolução, iniciamos diversos testes com base nos arquivos {\tt teste1-papi.c} e {\tt teste1\_ord-papi.c}, procurando entender o motivo de o primeiro programa ser aproximadamente 3 vezes mais lento que o segundo.

Analisando os resultados, as circunstâncias dos códigos e nossos conhecimentos até o momento, chegamos à conclusão que a diferença no tempo de execução dos programas era dado pelo acesso à variável {\tt count}. Assim, chegamos a algumas hipóteses para o posicionamento da variável relacionadas principalmente a caches, conforme segue.

\subsubsection{Número de falhas de cache}

As chamadas da variável {\tt count} eram feitas de forma consecutiva na execução do arquivo {\tt teste1\_ord-papi.c}, enquanto no arquivo {\tt teste1-papi.c} eram feitas mais esparsamente. Isso nos levou a pensar que poderia haver diferença significativa no número de falhas de acesso à memória cache.

Usamos os eventos {\tt PAPI\_L1\_DCM}, {\tt PAPI\_L2\_DCM} e {\tt PAPI\_L3\_DCM} para contar o número de falhas de cache no PAPI. Porém, o número foi bastante próximo para {\tt teste1.c} e {\tt teste1\_ord.c}, de forma que tivemos que abandonar essa hipótese.

\subsubsection{Número de acertos de instruções de cache}

Olhando os outros eventos disponíveis na biblioteca PAPI, encontramos os eventos {\tt PAPI\_L1\_ICH}, {\tt PAPI\_L2\_ICH} e {\tt PAPI\_L3\_ICH} que, ao invés de medirem as falhas de cache, medem o número de acertos de cache (\emph{cache hits}).

Esse contador revelou bastante diferença entre os programas {\tt teste1.c} e {\tt teste1\_ord.c}, como segue:

\begin{verbatim}
$ ./teste1-papi
Time: 21758132 Count 500264
Instruction cache hits (L1): 21480380
$ ./teste1_ord-papi
Time: 7742922 Count 500264
Instruction cache hits (L1): 7745845
\end{verbatim}

Com isso, concluímos que de fato são necessários muito menos acessos ao cache quando o vetor está ordenado.

\section{Parte 2}

A parte 2 do EP consiste em estimar o tamanho do cache e da linha baseado nos contadores de hardware que informam o número de falhas de cache.

\subsection{Utilizando hwloc}

Aceitamos a sugestão do enunciado do EP de usar \emph{hwloc}\footnote{http://www.open-mpi.org/software/hwloc/v1.7/} para obter os valores reais do tamanho do cache e da linha. Outras alternativas seriam ler {\tt /sys/devices/system/cpu/cpu*/*} ou ainda usar as rotinas que o PAPI no seu utilitário {\tt utils/mem\_info.c}, mas achamos interessante a promessa de um acréscimo na nota para quem usasse \emph{hwloc}.

Instalamos o programa através do gerenciador de pacotes do Ubuntu ({\tt apt-get install hwloc libhwloc-dev}). Também baixamos o código para usar como referência. Lendo o manual, descobrimos como inicializar o \emph{hwloc} e como obter os tamanhos de cache e linha:

\begin{verbatim}
/* Inicializa hwloc */
if (hwloc_topology_init(&topology)) {
    fprintf(stderr, "Erro em hwloc_topology_init\n");
    exit(1);
}
hwloc_topology_load(topology);

/* Pega tamanhos reais de cache e line */
level = 0;
for (obj = hwloc_get_obj_by_type(topology, HWLOC_OBJ_PU, 0);
        obj; obj = obj->parent) {
    if (obj->type == HWLOC_OBJ_CACHE) {
        real_cache[level] = obj->attr->cache.size;
        real_line[level] = obj->attr->cache.linesize;
        level++;
    }
}
numlevels = level;
\end{verbatim}

\subsection{Utilizando contadores do PAPI}

Havíamos aprendido a usar os contadores de hardware do PAPI na parte 1 do EP, quando formulamos a hipótese de que o programa {\tt teste1\_ord.c} seria mais rápido por causa do cache. Usamos aqui o mesmo código para contar o número de falhas de cache.

Para inicializar os contadores:

\begin{verbatim}
int events[3] = { PAPI_L1_DCM, PAPI_L2_DCM, PAPI_L3_DCM };
long long misses[3];
int papilevels = 3;

while (PAPI_start_counters(events, papilevels) != PAPI_OK) {
    papilevels--;
}
\end{verbatim}

E para lê-los:

\begin{verbatim}
if (PAPI_read_counters(misses, papilevels) != PAPI_OK) {
    fprintf(stderr, "Erro em PAPI_read_counters\n");
    exit(1);
}
printf("L1 cache miss: %lld\n", misses[0]);
printf("L2 cache miss: %lld\n", misses[1]);
printf("L3 cache miss: %lld\n", misses[2]);
\end{verbatim}

\subsection{Estimando tamanho do cache e da linha}

Não conseguimos fazer a parte do código para fazer a estimativa.

\section{Conclusões}

A enorme diferença de tempo de execução entre os programas {\tt teste1.c} e {\tt teste1\_ord.c} comprova empiricamente que, para além da complexidade dos algoritmos, a forma como o hardware funciona faz muita diferença na otimização dos programas.

O uso de cache nos processadores para evitar acesso à memória RAM faz com que os programas rodem muito mais rápido do que rodariam se precisassem sempre acessar a memória.

A biblioteca \emph{PAPI} pode ser uma ferramenta útil para detectarmos gargalos nas nossas implementações e entender em que hardware nossos códigos estão rodando.

O utilitário \emph{hwloc} pode nos ajudar a ter mais ideia da topologia dos nossos computadores.

\end{document}

